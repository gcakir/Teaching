\documentclass[a4paper]{article}

\usepackage[course={Secure and Dependable Systems},number=7,date=2017-05-10,duedate=2017-05-19,solutions]{../../myhomeworks}

\newcounter{chapter} % needed for dependencies of mylecturenotes
\usepackage[root=../..]{../../mylecturenotes}
\usepackage{../../macros/algorithm}

\begin{document}

\header

\begin{problem}{Asymmetric Encryption}{4}
Implement an asymmetric encryption scheme based on RSA.

It should have the following
\begin{compactitem}
 \item a key generation function that, given $n\in \N$, randomly chooses primes $p,q$ such that $p\cdot q\geq 2^n$, and then picks a random $e$ for which $d$ can be found,
 \item encryption and decryption functions that use RSA.
\end{compactitem}

Write a unit test that checks the inversion condition: pick an $n$ and an $n$-bit message, encrypt and decrypt it, and compare the result for equality.
\end{problem}

\begin{problem}{Hash Collisions}{4}
Consider the following (weak) hash function $hash:\{0,1\}^*\to \Z_N$ for $N=9993201131$: $hash(x)$ is obtained as follows
\begin{compactenum}
 \item append $0$s to $x$ such that its length is a multiple of $32$, and split the result into $32$-bit blocks $w_1,\ldots, w_n$
 \item put $h:=0$
 \item\footnote{The version I showed in the lecture used $h+w_i$ instead of $h+2+w_i$. I changed it to protect against attacks involving $w_1\in\{0,1\}$.}
  for each $i=1,\ldots,n$, put $h:=(h+2+w_i)^{1234567}\modop 9993201131$
 \item return $h$
\end{compactenum}

Using theory and/or brute force, find a collision of $hash$.
Show your work (theory and/or program).

\begin{solution}
This was a vague problem where multiple different attacks could be tried (like in a real-life attacks).

A good start was to recognize that $hash$ is similar to RSA and use a factorization attack.
This yields $N=p\cdot q$ with $p=99961$ and $q=99971$.
We can then obtain $g:x\mapsto x^{454586863}\modop 9993201131$ as the inverse of $f:x\mapsto x^{1234567}\modop 9993201131$.

Assume $h$ is the hash of some input.
For arbitrary $n$ and arbitrary $w_1,\ldots,w_{n-1}$, we can try to find a $w_n$ such that $hash(w_1\ldots w_n)=h$.
To do that, we put $h'=hash(w_1\ldots w_{n-1})$ and solve $h=f(h'+2+w_n)$ for $w_n$, i.e., $w_n=(g(h)-h'-2)\modop 9993201131$.
Because $2^{33}<9993201131<2^{34}$, there is a small chance that $w_n$ is too big to be a $32$-bit number.
In that case, we have to try again, e.g., with a different value for $w_{n-1}$.
That eventually yields a collision.

More generally, this construction allows inverting $hash$, which breaks it as a cryptographic hash function.
\end{solution}
\end{problem}

\begin{problem}{Password Hashing}{4}
Implement $hash$ from the previous problem as a function that hashes strings by using the ASCII codes of the characters as the values $w_1,\ldots,w_n$.

Assume $hash$ is (foolishly) used to hash passwords without any salting or stretching, and we expect to have access to some hashes in the future.
In order to prepare a break-in, build a table for pairs $(hash(s),s)$ for as many strings $s$ as you can so that you can lookup passwords once you have obtained the hashes.

You may work in groups to build larger tables.

\begin{solution}
Given that the previous solution allows constructing an inverse to $hash$, it is redundant for an attacker to table it.
But as an exercise the problem makes sense regardless.

To table $hash$, it is critical to invest into an efficient implementation of $hash$.
We should definitely use square-and-multiply for the exponentiation (divide-and-conquer!), and we can even use a precomputed binary representation of $1234567$.
Moreover, the modulus should be taken after each step, not just once at the end.

A number of further optimizations are possible.
Most importantly $hash(s_1\ldots s_n)$ can be obtained from $hash(s_1\ldots s_{n-1})$ in one step.
So if we have already tabled all hashes for strings of length $n-1$, we can reuse them to table all hashes for strings of length $n$ (dynamic programming!).

To run one iteration of the dynamic program, we can write a function $d(h)$ that takes a hash $h$ for a string $s$ and returns the hashes for the strings $sc$ for all characters $c$.
Because $d$ must be called on a large fixed set of values for $h$, it parallelizes with essentially no overhead.
If run with $k$ CPUs, the parallelized run time decreases by essentially $1/k$.
\end{solution}
\end{problem}


\begin{problem}{Bonus Problem}{depending on effort, at most 5\% of grade}
No deadline for this problem.

Using $hash$ from above, find a meaningful English word/sentence that hashes to $0$.

I have not checked how difficult this is.
If it is very easy, you have to find a very nice long sentence.
If it is very difficult, you may also look for pronounceable words that hash to small numbers.

\begin{solution}
Combining the ideas from the previous two problems may a good start here.
\end{solution}
\end{problem}
\end{document}