This chapter sketches some fundamental concepts from outside logic that are essential to make these notes self-contained.

Throughout this chapter, we fix an alphabet: a finite set $\Sigma$ of symbols.
Technically, it is perfectly sufficient to have a single symbol $\Sigma=\{0\}$.
However, it is more convenient to use at least two symbols such as in $\Sigma=\{0,1\}$ for theoretical analyses.
In practice, we use much bigger alphabets such as ASCII with $2^8$ symbols or Unicode with $2^{16}$ symbols (and more).

We look at the two fundamental concepts of computer science:
\begin{compactitem}
 \item the static part: sets of data,
 \item the dynamic part: functions between data.
\end{compactitem}

Both concepts are inherently infinitary.
Sets must be infinite to allow working with arbitrary data (e.g., the set of numbers or strings).
And functions on infinite sets must be infinite themselves because they must map infinitely many inputs.
But we can only ever actually compute with finite objects.
Therefore, we need effective representations of infinite sets and functions.

\section{Effective Representation of Functions}\label{sec:comp:algo}

\subsection{Computable Functions}\label{sec:comp:algo:def}

Consider a function $f:\Sigma^*\to\Sigma^*$ mapping inputs $i$ to outputs $o=f(i)$.
We want to define $f$ effectively, i.e., in a systematic way that allows other humans or machines to apply $f$ to $i$ and thus to compute $o$.

Therefore, we define:

\begin{definition}[Algorithm]\label{def:comp:algo}
An \textbf{algorithm} is a precisely described procedure, whose execution transforms words $i\in\Sigma^*$ to words $o\in\Sigma^*$ such that $o$ only depends on $i$. (In particular $o$ may not depend on when, where, and by whom the procedure is executed).

An algorithm $A$ \textbf{terminates for an input} $i$, if the execution finishes after finitely many steps.
In that case, we write $A(i)$ for the resulting output.
An algorithm $A$ \textbf{terminates}, if it terminates for all inputs.

Thus, every (terminating) algorithm induces a partial (total) function $A(-):\Sigma^*\to\Sigma^*$.
\end{definition}

Algorithms allow representing function systematically and reliably.
That makes them the basis of all mechanic computation.

\begin{terminology}
Instead of algorithm, people may also say \emph{(effective) method} or \emph{(effective) procedure}.
Here the word \emph{effective} emphasizes that the descriptions must be systematically executable (by a human or a machine).
\end{terminology}
 
\begin{remark}[Vagueness of the Definition of Algorithm]
Def.~\ref{def:comp:algo} is not precise: It uses the words ``precisely described procedure'' and ``execution'', which have not been defined previously.
That is intentional.
In retrospect, we can trace back the origin of computer science in the 1930s to this exact question: What is the description of a procedure, and what does it mean to execute it.

Nowadays, we have many different possible definitions.
In particularly, every single programming language yields a rigorous definition: The programs are the descriptions, and running a program is its execution.
We also call the former the syntax and the latter the semantics of the programming language.

Computer scientists have found many more possible definitions.
These may use very diverse formalisms such as Turing machines, grammars, or automata.
Even natural languages such as English or German are perfectly suitable for defining algorithms. (These descriptions are often called \emph{pseudo-code}.)
\end{remark}

Let us assume we have chosen one programming language or other formalism to make Def.~\ref{def:comp:algo} precise.
The next question is whether we can give an algorithm for every function $\Sigma^*\to \Sigma^*$.
We can immediately see that this is not the case: There are uncountably many such functions.
But whatever formalism we use to define \emph{algorithm} precisely, there can only be countably many algorithms.
The reason is that the algorithms must be described in some language (even it is English), and languages can only have countably many words.

We call the functions that do have algorithms \emph{computable}:

\begin{definition}[Computable Function]\label{def:comp:comp}
A partial function $f:\Sigma^*\to\Sigma^*$ is \textbf{computable} if there is an algorithm $A$ such that $A$ terminates for $i$ iff $f(i)$ is defined and then $A(i)=f(i)$.
\end{definition}

One might expect that different definitions of algorithm yield different computable functions.
This is not the case:

\begin{theorem}[Church-Turing Thesis]
Every rigorous definitions of algorithm makes the same functions computable.

Thus, the property of being computable is well-defined even though Def.~\ref{def:comp:algo} is vague.
\end{theorem}
\begin{proof}
Technically, we can never actually proof this because tomorrow someone might find a new definition of algorithm that is more powerful than everything we know.

But computer scientists have examined any number of possible definitions and have proved them all to be equivalent.
\end{proof}

Therefore, it does not matter which formalism we use to define algorithms.
For example, any programming language is fine.
In particular, whatever functions we can program in one programming language, we can also program in any other programming language.

\subsection{Generalizations}\label{sec:comp:algo:gen}

It is easy to generalize computability to more general functions by choosing special alphabets.

\paragraph{Different Alphabets}
A functions $f:\Sigma^\to\Sigma'^*$ that uses multiple alphabets can be seen as a function $f'$ that uses only the alphabet $\Sigma\cup\Sigma'$:
$f'(i)$ is defined whenever $i\in\Sigma^*$ and then $f'(i)=f(i)$.

Therefore, we also speak of computable functions if the involved alphabets are different.

\paragraph{Multiple Arguments}
An $n$-ary function $f:\Sigma^*\times\ldots\Sigma^*\to\Sigma^*$ can be seen as a unary function $f'$ on $\Sigma\cup\{|\}$: We define $f'(i_1|\ldots|i_n)=f(i_1,\ldots,i_n)$.

Therefore, we also speak of computable $n$-ary functions.

\paragraph{Concrete Sets}
Most mathematically relevant countable sets can be written as $\Sigma^*$ (or appropriate subsets).
For example, we can consider $\N$ to be the set $\{|\}^*$ or an appropriate subset of $\{0,1\}^*$ or of $\{0,1,2,3,4,5,6,7,8,9\}^*$.
Therefore, we also speak of computable functions on the natural numbers.

Computable functions involving $\Z$ and $\Q$ are obtained similarly by adding symbols for $-$ and $/$.

We can never represent compute with all of $\R$ or $\C$ though because they are not countable.

\subsection{Closure Properties}\label{sec:comp:algo:prop}

Most practically relevant functions are computable:

\begin{theorem}\label{thm:comp:comp}
The following functions are computable:
\begin{compactitem}
\item the identity function $x\mapsto x$
\item for every $c\in\Sigma^*$, the constant function $x\mapsto c$
\item for computable functions $f,g$, the composition $x\mapsto f(g(x))$
\item for computable functions $f:\Sigma^*\to \{1,\ldots,n\}$ and $g_1,\ldots,g_n:\Sigma^*\to\Sigma^*$ the case-based function
\[x\mapsto \cas{g_1(x) \mifc f(x)=1 \\ \vdots\\ g_n(x) \mifc f(x)=n}\]
\item for computable functions $\mathrm{case}:\Sigma^*\to \{0,1\}$, $\mathrm{base}:\Sigma^*\to\Sigma^*$, $\mathrm{next}:\Sigma^*\to\Sigma^*$, and $\mathrm{step}:\Sigma^*\times\Sigma^*\to\Sigma^*$ the recursive function $r$ given by the following program
 \[\mathtt{define}\; r(x) := \mathtt{if}\; \mathrm{case}(x)=0\; \mathtt{then}\; \mathrm{base}(x) \;\mathtt{else}\; \mathrm{step}(x,r(\mathrm{next}(x)))\]
\end{compactitem}
\end{theorem}
\begin{proof}
It is easy to implement these functions in any programming language.
\end{proof}

\begin{theorem}\label{thm:comp:comp2}
All computable functions can be obtained by combining the operations from Thm.~\ref{thm:comp:comp}.
\end{theorem}
\begin{proof}
We only sketch the proof.

First, we build a minimal programming language $M$ out of the operations: identity corresponds to variables $x$, constants to values $c\in \Sigma^*$, composition to sequential operation $;$, case-based function to case-split using an $\mathtt{if}$ or $\mathtt{switch}$ command, and recursion to recursive functions.

Second, we show that every programming language $P$ can be reduced to $M$.
We can do that by implementing an interpreter for $P$ in $M$.
(Or, more realistically, we proceed step-wise: We implement $M_2$ in $M$, and $M_3$ in $M_2$, and so on until we reach $P$.)
\end{proof}


\section{Effective Representation of Sets}\label{sec:comp:dec}

\subsection{Decidable and Enumerable Sets}\label{sec:comp:data:def}

We can easily represent the set $\Sigma^*$: It is the set of words over $\Sigma$, i.e., the set of sequences of symbols.
In a programming language, we usually call them strings.
But we almost never need $\Sigma^*$ --- we usually need certain subsets of $\Sigma^*$.
For example, the set of all well-formed $\FOL$-formulas is a subset of $\Sigma^*$ for $\Sigma=\{\wedge,\vee,\impl,\neg,\forall,\exists\}\cup\{a,\ldots,z\}\cup\{(,)\}\cup\{,\}$.
Similarly, every programming language is a subset of the set of all strings.

A set is defined by its elements.
But there are two natural ways to represent effectively what the elements of a set are:

\begin{definition}[Decidable Sets]\label{def:comp:dec}
A set $S\sq\Sigma^*$ is called \textbf{decidable} if there is a computable total function $f$ such that $f(i)=1$ if $i\in S$ and $f(i)=0$ otherwise.
(Here, $0$ and $1$ are two arbitrary different elements of $\Sigma^*$ serving as truth values.)
\end{definition}

\begin{definition}[Enumerable Sets]\label{def:comp:enum}
A set $S\sq\Sigma^*$ is called \textbf{enumerable} if there is a computable total function $f$ such that $\mathrm{image}(f)=S$.
\end{definition}

If $S$ is decidable, we have an effective way to test for being an element of $S$: We can execute $f(i)$ to determine whether $i\in S$ or not.
We also call $f$ a decision procedure for $S$.

If $S$ is enumerable, we have an effective way to list all the elements of $S$: We can run $f$ on all possible inputs and list all the outputs.
We also call $f$ an enumeration of $S$.

\begin{terminology}
Instead of enumerable, people may also say \emph{recursively enumerable} or \emph{semi-decidable}.
(See Thm.~\ref{thm:comp:enumdec} and~\ref{thm:comp:enum} to understand why \emph{semi-decidable} makes sense.)

Note that \emph{denumerable} (also called countable) is different from enumerable.%
\footnote{In German, these words are also confusing: denumerable is \emph{abz\"ahlbar} and enumerable is \emph{aufz\"ahlbar}.}
A set $S$ is denumerable if there is \emph{some} function with domain $\N$ and image $S$; a set is enumerable if there is a \emph{computable} function with domain $\N$ and image $S$.
In particular, all subsets of $\Sigma^*$ are denumerable.
\end{terminology}

\subsection{Generalizations}\label{sec:comp:data:gen}

Like in Sect.~\ref{sec:comp:algo:gen}, we extend the concepts to all sets that can be represented using an appropriate alphabet.
Thus, we speak of decidable and enumerable subsets of $\N$, $\Z$, and $\Q$.

It is easy to represent Cartesian products of sets by using an alphabet $\Sigma\cup\{|\}$.
Therefore, we also speak of decidable and enumerable subsets of Cartesian products.

An important special case is the set of computable functions.
Because the computable functions can be represented as programs in some programming language, we can see each computable function as an element of $\Sigma^*$ where $\Sigma$ is the alphabet of the programming language.
Consequently, we can speak of decidable and enumerable sets of computable functions.

\subsection{Closure Properties}\label{sec:comp:data:prop}

Enumerable and decidable are not equivalent.
Being able to decide is stronger because it lets us enumerate both the elements that are in $S$ and those that are not in $S$:

\begin{theorem}\label{thm:comp:enumdec}
A set $S$ is decidable iff both $S$ and $\Sigma^*\sm S$ are enumerable.
\end{theorem}
\begin{proof}
Left-to-right: Let $f$ be a decision procedure for $S$.
To enumerate $S$, we return $i$ iff $f(i)=1$ and nothing otherwise.
To enumerate $\Sigma^*\sm S$, we return $i$ iff $f(i)=0$ and nothing otherwise.

Right-to-left: Let $f^+$ and $f^-$ be enumerations for $S$ and $\Sigma^*\sm S$, respectively.
Let $x_1,x_2,\ldots$ be all the elements of $\Sigma^*$.
To decide $S$, we take input $i$ and compute $f^+(x_1),f^-(x_1),f^+(x_2),f^-(x_2),\ldots$ until we find $i$. We return $1$ or $0$ depending on whether we found $i$ in $f^+$ or in $f^-$.
\end{proof}

If a set is enumerable, it is not necessarily decidable.
Given an enumeration $f^+$, we might try to decide $i\in S$ by computing $f^+(x_1),f^+(x_2),\ldots$.
If we find $i$ in this sequence, we can return $1$.
But it would take infinitely long to determine that $i$ is not in this sequence.
Therefore, we can never return $0$.
The following theorem makes this more precise:

\begin{theorem}\label{thm:comp:enum}
The following are equivalent:
\begin{compactitem}
 \item There is a total computable function $f$ with whose image is $S$ (i.e., $S$ is enumerable).
 \item There is a partial computable function $f$ with whose image is $S$.
 \item There is a partial computable function $f$ such that $f(i)=1$ iff $i\in S$ and $f(i)$ is undefined otherwise.
\end{compactitem}
\end{theorem}
\begin{proof}
It is straightforward to give the necessary enumerations.
\end{proof}


\begin{theorem}[Closure Properties of Decidable Sets]
The following sets are decidable:
\begin{compactitem}
  \item $\es$ and $\Sigma^*$
  \item all finite or cofinite\footnote{A set is cofinite if it contains all but finitely many elements.} sets
  \item the intersection of decidable sets
  \item the union of decidable sets
  \item the complement of a decidable set
\end{compactitem}
\end{theorem}
\begin{proof}
It is straightforward to give the necessary decision procedures.
\end{proof}

\begin{theorem}[Closure Properties of Enumerable Sets]
The following sets are enumerable:
\begin{compactitem}
  \item $\es$ and $\Sigma^*$
  \item all finite or cofinite sets
  \item the union of enumerable sets
\end{compactitem}
\end{theorem}
\begin{proof}
It is straightforward to give the necessary enumerations.
\end{proof}

It is easy to understand why the enumerable sets are not closed under intersection and complement.
If we tried to enumerate them, we would run into the problem that we have to wait infinitely long to conclude anything.

Note that neither decidable nor enumerable sets are closed under subsets.
For example, a subset of a decidable set may be undecidable.

\subsection{Counter-Examples}

Some of the deepest results in computer science are about demonstrating that certain sets are not decidable or not enumerable.

The most important ones are about computation itself:
\begin{itemize}
\item Halting problem: Consider pairs $(f,i)$ where $f$ is a computable function and $i$ an input for $f$.
The set of all pairs $(f,i)$ such that $f(i)$ is defined (i.e., such that computation of $f(i)$ terminates) is enumerable but not decidable.
\item The set of computable functions $f$ such that $f(i)$ is defined for all inputs (i.e., the set of terminating programs) is not enumerable.
\end{itemize}

In logic, we also encounter some critical examples:
\begin{itemize}
\item The set of theorems over a $\FOLEQ$-theory is enumerable but not necessarily decidable.
For some special theories, decidability holds, but this is a very rare situation and usually non-trivial to prove.\\
\item The set of consistent theories of $\FOLEQ$ is undecidable.
\item The same applies to virtually any other logic with the exception of $\PL$.
For $\PL$, consistency and theoremhood are decidable.
\item The set of true sentences over the natural numbers is not enumerable. (See Thm.~\ref{thm:natincomplete}.)
\item The same applies to most other interesting models.
\end{itemize}
