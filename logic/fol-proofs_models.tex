\section{An Abstract Definition of a Logic}

\begin{definition}[Logic]\label{def:logic:abs}
A \emph{logic} consists of
\begin{itemize}
	\item a collection of signatures $\Sig$,
	\item a set of sentences $\Sen(\Sigma)$ for every $\Sigma\in\Sig$,
	\item a collection of models $\Mod(\Sigma)$ for every $\Sigma\in\Sig$,
	\item	a model-theoretic definition of truth and consequence: for every $\Sigma\in\Sig$, a relation $\md_\Sigma$ between $\Mod(\Sigma)$ and $\Sen(\Sigma)$,
	\item an inference system $\Pf(\Sigma)$ for every $\Sigma\in\Sig$,
	\item a proof theoretic definition of consequence: for every $\Sigma\in\Sig$, a judgment $\Delta\der_\Sigma F$.
\end{itemize}
\end{definition}

Thus, we have two alternative ways of defining theorems and consequence: model and proof theoretically.
The former was strongly influenced by Tarski \cite{tarskilogic} and \cite{robinsonmodeltheory} and has become the dominant method in mathematical logic.
The latter was strongly influenced by Hilbert's program \cite{hilbertsprogram} and the work by G\"odel \cite{goedelcompleteness} and Gentzen \cite{gentzen}.
It has become important in computer science because it permits using computers to check and search for proofs.

\section{Soundness and Completeness}

Proof and model theory work best when used in combination: We use theories as interfaces, models as interface providers, and the proofs of a theory can be reused in every model.
This works if P-consequence and M-consequence are the same, which is where soundness and completeness come in.

\subsection{Definitions}

We define:
\begin{definition}[Soundness and Completeness]
A \emph{logic} is \emph{sound} if for all signatures $\Sigma$, all sets of sentences $\Delta$ and all sentences $F$,
\[\Delta\der_\Sigma F \tb\mimplies\tb \moda{\Delta}{\Sigma}{F}.\]

A \emph{logic} is \emph{complete} if for all signatures $\Sigma$, all sets of sentences $\Delta$ and all sentences $F$,
\[\moda{\Delta}{\Sigma}{F} \tb\mimplies\tb \Delta\der_\Sigma F.\]
\end{definition}

In other words, it is complete if M-consequence implies P-consequence; or if all M-theorems are P-theorems.
And it is sound if the implication goes the other way around.

\begin{notation}
Note that both theories and logics can be complete. The two notions are only loosely related and should be seen as different notions that happen to share the name ``complete''.
\end{notation}

Soundness and completeness extend to the other notions mentioned in Not.~\ref{not:mod} and~\ref{not:pf}:
\begin{theorem}\label{thm:pfmod:mp}
If a logic is sound then,
\begin{itemize}
	\item P-contradictions are M-contradictions,
	\item P-inconsistent theories are M-inconsistent,
	\item P-complete theories are M-complete.
\end{itemize}
For completeness, the opposite implications hold.
\end{theorem}
\begin{proof}
Straightforward because all notions are defined in terms of the consequence relation.
\end{proof}

Note that Thm.~\ref{thm:pfmod:mp} does not include a case for ``consistent'' theories. For ``consistent'', the implication is flipped because ``M-consistent theories are P-consistent'' is the same as ``not M-inconsistent theories are not P-inconsistent'' and that is equivalent to ``P-inconsistent theories are M-inconsistent''. More precisely, we have the following:

\subsection{Intuitions}

Soundness means that the proof theory is correct: If we can prove a consequence, it holds about all models.
It is usually easy to show soundness by induction on proof trees.

Completeness is more complicated (and there are useful logics that are not complete).
To understand it better, we show the following:

\begin{lemma}\label{lem:pfmod:complete}
The following are equivalent for any logic $L$ that has a negation together with rules for negation introduction and double-negation elimination (i.e., $\neg \neg F\der F$):
\begin{enumerate}
 \item $L$ is complete (i.e., M-consequence implies P-consequence).
 \item Every P-consistent theory has a model (i.e., P-consistency implies M-consistency).
\end{enumerate}
\end{lemma}
\begin{proof}
By circular implication: \\
1 implies 2: See Thm.~\ref{thm:pfmod:mp}.
2 implies 1: To show completeness, assume a theory $(\Sigma;\Theta)$ and an M-theorem $F$ (*). Then $(\Sigma;\Theta\cup\{\neg F\})$ has no model. Using 2, we obtain that $(\Sigma;\Theta\cup\{\neg F\})$ is P-inconsistent. Using negation introduction, we can derive $\neg \neg F$ from $\Theta$, and because of double-negation elimination, we can also derive $F$. Thus, $F$ is a P-theorem of $(\Sigma;\Theta)$.
\end{proof}

Lem.~\ref{lem:pfmod:complete} is the typical way how we prove completeness.
\medskip

Then we have two intuitions for the word ``completeness''.
If the model theory defines the theorems, completeness is a property of the proof theory: A logic is complete if it has enough proofs to derive all the theorems.

If the proof theory defines the theorems, completeness is a property of the model theory: A logic is complete if it has enough models to interpret all the consistent theories.
\medskip

Another way to think about soundness and completeness is to imagine a step-by-step logic design.
Say initially we have no proof rules and no models.
Then no formula is a P-theorem, and all formulas are M-theorems.
The logic is as sound as it can be but also as far from complete as it can be.

We can change the logic by adding proof rules (which increases the set of P-theorems) and/or by adding models (which decreases the set of M-theorems).
If we have added enough proof rules and models, the sets of P-theorems and M-theorems meet, and we have a sound and complete logic.
If we keep adding proof rules or models, we will have too many: The set of P-theorems becomes bigger than the set of M-theorems, and we lose soundness (while staying complete).

Thus, to make a logic sound, we have to remove rules and/or models.
And to make a logic complete, we have to add rules and/or models.

\begin{example}[Empty Universe]
As pointed in Rem.~\ref{rem:mod:empty} and~\ref{rem:pf:empty}, there is flexibility as to how to treat the empty universe.

If we use the usual rules for first-order logic with all models (including the empty universe), the logic is not sound.
To make it sound, we have two options:
\begin{itemize}
  \item remove models, specifically the one with the empty universe: This is the option chosen by most logic textbooks.
  \item remove (or weaken) rules: This is the option chosen in these notes by using weakening the $\forall E$ rule.
\end{itemize}
\end{example}

% similar: $\vee R$ rule of sequent calculus, which allows an assumption to escape its scope
% classical solution: upgrade the bug to a feature

\subsection{Compactness}

Compactness is the property that all consequences are caused by finitely many axioms:

\begin{definition}
A model theory is called \defemph{compact} if $\moda{\Sigma}{\Theta}{F}$ implies $\moda{\Sigma}{\Theta'}{F}$ for some finite set $\Theta'\sq\Theta$.
\end{definition}

\begin{theorem}
If a logic is sound and complete, then it is compact.
\end{theorem}
\begin{proof}
We always have that $\iscons{\Sigma}{\Theta}{}{F}$ implies $\iscons{\Sigma}{\Theta'}{}{F}$ for some finite set $\Theta'\sq\Theta$ because a proof can only use finitely many axioms from $\Theta$.

Therefore, soundness and completeness together imply compactness.
\end{proof}

\section{The Big Theorems of First-Order Logic}

The most important theorems about FOL and FOLEQ are soundness and completeness.
The first sound and complete calculi for first-order logic were given by G\"odel \cite{goedelcompleteness} and Gentzen \cite{gentzen} in the 1930s.

\subsection{Soundness}

\begin{theorem}[Soundness of FOL]
FOL is sound.
\end{theorem}
\begin{proof}
By induction on derivations.
The theorem only mentions those judgments where $\Gamma=\es$ (i.e., $\Gamma$ is omitted). This is as usual the only case we are really interested in. But in the induction we will encounter all judgments.

Therefore, we have to prove something more general, namely:
\begin{quote}
$\Gamma;\Delta\der_\Sigma F$ implies that \\
 \tb for every model $I\in\Mod(\Sigma)$ and every assignment $\alpha$ for $\Gamma$, \\
 \tb we have that\\
 \tb\tb if $\moda{I,\alpha}{\Sigma}{T}$ for all $T\in\Delta$,\\
 \tb\tb then $\moda{I,\alpha}{\Sigma}{F}$.
\end{quote}
Note that if we put $\Gamma=\es$, this is the same as saying $\moda{\Delta}{\Sigma}{F}$.

Now we can prove this more general statement by induction on derivations. The intuition behind this is the following: For all axioms $\rul{\der A}{}{}$, show that $A$ is true in every model. For every rule $\rul{J}{J_1\tb\ldots\tb J_n}{}$ show that applying the rule preserves truth, i.e., show the above for $J$ assuming it is true (induction hypothesis) for the $J_i$.
Since proof trees are built up by composing rules, the induction requires one case for every rule. Therefore, we often say that a single rule is sound: The soundness of a rule $R$ means that the induction step succeeds for $R$; then to show soundness of the logic, we need to show that every rule is sound.

In the following we give three example soundness proofs for the rules $Axiom$, $\wedge I$, and $\arr I$.
\medskip

\noindent The case for $Axiom$. For this rule, there are no hypotheses. So we have to show that for every model and assignment $I$ and $\alpha$, if $\moda{I,\alpha}{\Sigma}{T}$ for all $T\in\Delta$, then $\moda{I,\alpha}{\Sigma}{A}$. This is trivial because $A\in\Delta$.
\medskip

\noindent The case for $\wedge I$. For this rule, there are two hypotheses. Applying the induction hypothesis to them yields:
\begin{itemize}
\item (1) For every model and assignment $I$ and $\alpha$, if $\moda{I,\alpha}{\Sigma}{T}$ for all $T\in\Delta$, then $\moda{I,\alpha}{\Sigma}{A}$.
\item (2) For every model and assignment $I$ and $\alpha$, if $\moda{I,\alpha}{\Sigma}{T}$ for all $T\in\Delta$, then $\moda{I,\alpha}{\Sigma}{B}$.
\end{itemize}
We have to show: For every model and assignment $I$ and $\alpha$, if $\moda{I,\alpha}{\Sigma}{T}$ for all $T\in\Delta$, then $\moda{I,\alpha}{\Sigma}{A\wedge B}$. This is obvious. But to be totally clear, here is the full proof:
\begin{itemize}
\item We pick an arbitrary model $I_0$ and an arbitrary assignment $\alpha_0$.
\item We assume that (3) $\moda{I_0,\alpha_0}{\Sigma}{T}$ for all $T\in\Delta$. We have to show $\moda{I_0,\alpha_0}{\Sigma}{A\wedge B}$.
\begin{itemize}
\item From (3), using (1) and (2), we obtain (4) $\moda{I_0,\alpha_0}{\Sigma}{A}$ and (5) $\moda{I_0,\alpha_0}{\Sigma}{B}$, respectively.
\item From (4) and (5), using the definition of the interpretation of formulas, we obtain $\moda{I_0,\alpha_0}{\Sigma}{A\wedge B}$.
\item Done.
\end{itemize}
\item Done.
\end{itemize}
\medskip

\noindent The case for $\arr I$. For this rule, there is one hypothesis. Applying the induction hypothesis to it yields:
\begin{itemize}
\item (1) For every model and assignment $I$ and $\alpha$, if $\moda{I,\alpha}{\Sigma}{T}$ for all $T\in\Delta\cup\{A\}$, then $\moda{I,\alpha}{\Sigma}{B}$.
\end{itemize}
We have to show: For every model and assignment $I$ and $\alpha$, if $\moda{I,\alpha}{\Sigma}{T}$ for all $T\in\Delta$, then $\moda{I,\alpha}{\Sigma}{A\arr B}$.
\begin{itemize}
\item We pick an arbitrary model $I_0$ and an arbitrary assignment $\alpha_0$.
\item We assume that (2) $\moda{I_0,\alpha_0}{\Sigma}{T}$ for all $T\in\Delta$. We have to show $\moda{I_0,\alpha_0}{\Sigma}{A\arr B}$.
\item We distinguish two cases: $\moda{I_0,\alpha_0}{\Sigma}{A}$ and $\nmoda{I_0,\alpha_0}{\Sigma}{A}$.
\begin{itemize}
\item $\nmoda{I_0,\alpha_0}{\Sigma}{A}$. Then by definition of the interpretation of formulas $\nmoda{I_0,\alpha_0}{\Sigma}{A\arr B}$. Done.
\item (3) $\moda{I_0,\alpha_0}{\Sigma}{A}$.
  \begin{itemize}
    \item Combining (2) and (3), we are able to apply (1), from which we obtain (4) $\moda{I_0,\alpha_0}{\Sigma}{B}$.
    \item From (3) and (4), using the definition of the interpretation of formulas, we obtain $\moda{I_0,\alpha_0}{\Sigma}{A\arr B}$.
    \item Done.
  \end{itemize}
\end{itemize}
\item Done.
\end{itemize}
\end{proof}

\subsection{Completeness}

The completeness of FOL requires a few additional constructions.

The central idea of the completeness proof is to build a model $S$ of a theory $(\Sigma;\Theta)$ from the terms and proofs.
First, we define the interpretation of function symbols -- this is called the term model:

\begin{definition}[Term Model]\label{def:termmodel}
Let $\Sigma$ be a signature without predicate symbols.

Then we define the model $S$ as follows:
 \begin{itemize}
  \item $\TERM^S=\{t\;|\;\istermS{}{t}\}$
  \item $f^S(u_1,\ldots,u_n)=f(u_1,\ldots,u_n)$ for $n=\arit(f)$
 \end{itemize}
\end{definition}

\begin{remark}
If we used the variant of $\FOLEQ$ that excludes the empty universe, we might have to add some nullary function symbol to $\Sigma$ to make sure that the universe of the term model is not empty.
\end{remark}

Read the last line in Def.~\ref{def:termmodel} very carefully:
\begin{enumerate}
 \item $\TERM^S$ is a set, which $S$ defines to be set of terms.
 \item $f$ is a function symbol.
 \item $f^S$ is an $n$-ary function on $\TERM^S$, i.e., a function that takes terms and returns a term.
 \item $f^S(u_1,\ldots,u_n)$ is the application of $f^S$ to some arguments, which in this case must be terms.
 \item $f(u_1,\ldots,u_n)$ is the term formed from the function symbol $f$ and the term $u_i$.
\end{enumerate}

The important property of the term model is that it interprets every term as itself:
\begin{lemma}\label{thm:termmodel}
In the situation of Def.~\ref{def:termmodel}, we have
  \[\semm{t}{S}=t\]
for all terms $\istermS{}{t}$.
\end{lemma}
\begin{proof}
We use induction on $t$:
\begin{itemize}
  \item Case $t=f(t_1,\ldots,t_n)$: We have
   \[\semm{f(t_1,\ldots,t_n)}{S}=^{Def}f^S(\semm{t_1}{S},\ldots,\semm{t_n}{S})=^{IH}f^S(t_1,\ldots,t_n)=^{Def}f(t_1,\ldots,t_n)\]
  \item Case $t=x$: impossible because $t$ has no free variables.
\end{itemize}
\end{proof}

You should read the proof very carefully, too, and understand how the three steps in the case for function terms proceed.

We cannot interpret every formula as itself because the set $\FORM^S$ is already fixed to be $\{0,1\}$.
But we can do something similar: We interpret a formula $F$ as $1$ if it is a theorem and as $0$ if is a contradiction.
But that does not quite work immediately: If $\Theta$ is not complete, there are formulas that are neither theorems nor contradictions.
Therefore, we show:

\begin{lemma}\label{thm:complete_extension}
For every P-consistent theory $\Theta$, there is a P-consistent complete theory $\ov{\Theta}$ with $\Theta\sq\ov{\Theta}$.
\end{lemma}
\begin{proof}
Let $F_1,F_2,\ldots,$ be an enumeration of all $\Sigma$-sentences (in any order).
We define theories $\Theta_0,\Theta_1,\Theta_2,\ldots$ inductively:
\begin{itemize}
  \item $\Theta_0=\Theta$
  \item $\Theta_{n+1}= \cas{(a)\tb \Theta_n\cup \{F_{n+1}\} \mifc \mnot \iscons{\Sigma}{\Theta_n}{}{\neg F_{n+1}}\\ (b) \tb \Theta_n \mothw }$
\end{itemize}
We define $\ov{\Theta}=\bigcup_{i\in\N}\Theta_i$.
\medskip

We have to show three properties about $\ov{\Theta}$.

\begin{itemize}
\item $\Theta\sq\ov{\Theta}$: That is obvious.
\item Completeness: Consider a sentence $F$. We know $F=F_n$ for some $n$. By construction $\ov{\Theta}$ contains $F_n$ for every sentence or can prove $\neg F_{n+1}$.
Thus it can prove $F_n$ or $\neg F_n$.
\item P-consistency:
\begin{enumerate}
  \item We show that if $\Theta_{n+1}$ is P-inconsistent, then so is $\Theta_n$.
  Assume $\Theta_{n+1}$ is $P$-inconsistent.
        We distinguish the two cases in the definition of $\Theta_{n+1}$:
        \begin{itemize}
          \item[(a)] Then we have $\iscons{\Sigma}{\Theta_n,F_{n+1}}{}{\false}$ and thus (using the negation introduction rule) $\iscons{\Sigma}{\Theta}{}{\neg F_{n+1}}$. But that violates the assumption of case (a).
        There are two cases for $G$:
          \item[(b)] Then $\Theta_{n+1}=\Theta_n$, so $\Theta_n$ is also inconsistent.
        \end{itemize}
     \item Thus, because $\Theta_0$ is P-consistent, so are all $\Theta_n$.
     \item If $\ov{\Theta}$ were P-inconsistent, there would be a proof $p$ of $\iscons{\Sigma}{\ov{\Theta}}{}{\false}$.
       Because a proof can only use finitely many axioms, there would be a $\Theta_n$ that contains all axioms used in $p$.
       Then $\Theta_n$ would be inconsistent, too, which is impossible.
   \end{enumerate}
\end{itemize}
\end{proof}

Then we can extend Def.~\ref{def:termmodel}:

\begin{definition}[Term Model with Predicates]\label{def:termmodel2}
For a theory $(\Sigma;\Theta)$, we define the model $S$ as follows:
 \begin{itemize}
  \item $\TERM^S$ and $f^S$ are as in Def.~\ref{def:termmodel}.
  \item $p^S(u_1,\ldots,u_n)=\cas{1 \mifc \iscons{\Sigma}{\Theta}{}{p(u_1,\ldots,u_n)} \\
                                  0 \mothw}$
 \end{itemize}
\end{definition}

Correspondingly, we can extend Def.~\ref{thm:termmodel}:
\begin{lemma}\label{thm:termmodel2}
In the situation of Def.~\ref{def:termmodel2} let $\Theta$ be complete and consistent.
Then, we have
  \[\semm{F}{S}=\cas{1 \mifc \iscons{\Sigma}{\Theta}{}{F} \\
                     0 \mifc \iscons{\Sigma}{\Theta}{}{\neg F}}\]
for all sentences $\isformS{}{F}$ that do not use $\forall$, $\exists$, or $\doteq$.
\end{lemma}
\begin{proof}
We use induction on $F$:
\begin{itemize}
  \item Case $F=p(t_1,\ldots,t_n)$: We have
   \[\semm{p(t_1,\ldots,t_n)}{S}=^{Def}p^S(\semm{t_1}{S},\ldots,\semm{t_n}{S})=^{\text{Lem.}~\ref{thm:termmodel}}p^S(t_1,\ldots,t_n)
    =^{Def}\cas{1 \mifc \iscons{\Sigma}{\Theta}{}{p(u_1,\ldots,u_n)} \\
                0 \mifc \iscons{\Sigma}{\Theta}{}{\neg p(u_1,\ldots,u_n)}}\]
   Here, the last step uses the fact that $\Theta$ is complete to turn ``otherwise'' into ``if $\iscons{\Sigma}{\Theta}{}{\neg p(u_1,\ldots,u_n)}$''.
  \item Case $F=G\wedge H$: We distinguish two cases:
    \begin{itemize}
      \item $\semm{G}{S}=1$ and $\semm{H}{S}=1$: We evaluate both sides of the equation:
       \begin{itemize}
         \item[(left)] We get $\semm{G\wedge H}{S}=1$.
         \item[(right)] The induction hypothesis for $G$ and $H$ yields $\iscons{\Sigma}{\Theta}{}{G}$ and $\iscons{\Sigma}{\Theta}{}{H}$.
            Thus, (using the conjunction introduction rule), we have $\iscons{\Sigma}{\Theta}{}{G\wedge H}$.
            Thus, the right side also yields $1$.
       \end{itemize}
      \item $\semm{G}{S}=0$ or $\semm{H}{S}=0$: We evaluate both sides of the equation:
        \begin{itemize}
         \item[(left)] We get $\semm{G\wedge H}{S}=0$.
         \item[(right)] The induction hypothesis for $G$ or $H$ yields $\iscons{\Sigma}{\Theta}{}{\neg G}$ or $\iscons{\Sigma}{\Theta}{}{\neg H}$.
            Thus, (using the conjunction elimination rules and negation introduction+elimination rules), we have $\iscons{\Sigma}{\Theta}{}{\neg(G\wedge H)}$.
            Thus, the right side also yields $0$.
        \end{itemize}
    \end{itemize}
  \item The cases for negation, disjunction, and implication proceed like the one for conjunction.
\end{itemize}
\end{proof}

\begin{remark}
There is a more complex version of Lem.~\ref{thm:termmodel2} that applies to all sentences.
However, all of the above definitions have to be changed to use a different set $\TERM^S$.
The key ideas is
 \begin{itemize}
   \item To handle equality, $\TERM^S$ is the quotient of the set of terms modulo the equivalence relation $\iscons{\Sigma}{\Theta}{}{t_1\doteq t_2}$.
   This is necessary to make sure $\semm{t_1}{S}=\semm{t_2}{S}=1$ whenever $\iscons{\Sigma}{\Theta}{}{t_1\doteq t_2}$.
   \item To handle quantifiers, $\Sigma$ and $\Theta$ have to be extended: If we can prove $\forall x_1\ldots\forall x_n\;\exists x A$, we have to add an $n$-ary function symbols $f$ and an axiom $\forall x_1\ldots\forall x_n\;A[x/f(x_1,\ldots,x_n)$.
    This is necessary to make sure that $\TERM^S$ contains the element necessary for $\semm{\exists x \;A}=1$.
 \end{itemize}

%There are also other versions of Lem.~\ref{thm:termmodel2} that use the same $\TERM^S$. For example, we can permit $\forall$ if $\neg$ may only occur before atomic formulas.
%For Horn theories, the Herbrand model always exists. (In a Horn theory, all formulas in $\Theta$ must be of the form $\forall x_1\;\ldots\;\forall x_m\;((A_1\wedge \ldots A_n)\arr A)$ for atoms $A_1,\ldots,A_n,A$. Horn theories are already in Skolem normal form.)
%\item The Herbrand model of Horn theories is the base of the Prolog programming language. Prolog interprets a command $A:-A_1,\ldots,A_m$ with free variables $x_1,\ldots,x_n$ as the axiom $\forall x_1\;\ldots\;\forall x_m\;((A_1\wedge \ldots A_n)\arr A)$. A query $A$ (without free variables) is answered with Yes or No depending on whether $A$ holds in the Herbrand model constructed from all commands of the current program.
\end{remark}

Finally, we can prove completeness of FOL:

\begin{theorem}[Completeness of FOL]
FOL is complete.
\end{theorem}
\begin{proof}
We only prove the completeness of FOL with $\forall$, $\exists$, and $\doteq$ removed.
The general prove proceeds similarly after generalizing Lem.~\ref{thm:termmodel2}.

By Lem.~\ref{lem:pfmod:complete}, we only have to show that every P-consistent FOL theory has a model.
So assume a P-consistent theory $(\Sigma;\Theta)$.

Using Lem.~\ref{thm:complete_extension}, we can assume that $\Theta$ is complete.
Otherwise, we construct a model of $\ov{\Theta}$, which will also be a model of $\Theta$.

We use the term model from Def.~\ref{def:termmodel2}.
By Lem.~\ref{thm:termmodel2}, it satisfies all formulas in $\Theta$.
\end{proof}

%
%\begin{theorem}
%Assume $(\Sigma;\Theta)$ is in Skolem normal form, i.e., all sentences in $\Theta$ are in prenex normal form and there are no existential quantifiers. Then $\Mod(\Sigma;\Theta)\neq\es$ iff $H\in\Mod(\Sigma;\Theta)$.
%\end{theorem}
%\begin{proof}
%Omitted.
%\end{proof}
%Intuitively: If there are any models, then the Herbrand model is one of them.
%
%More elaborately, assume we have a theory $(\Sigma;\Theta)$. We know we can transform it (by the clausification algorithm from above) into a theory $(\Sigma';\Theta')$ in Skolem normal form. Furthermore, we know that $(\Sigma;\Theta)$ is consistent iff $(\Sigma';\Theta';)$ is. We can construct the Herbrand model $H$ for $(\Sigma';\Theta')$. Then we can check whether $H\in\Mod(\Sigma';\Theta')$, i.e., we check whether $\moda{\Theta'}{\Sigma'}{F}$ for all $F\in\Theta'$. If this is the case, we have found a model of $(\Sigma';\Theta')$ and therefore $(\Sigma';\Theta')$ and $(\Sigma;\Theta)$ are consistent. If this is not the case, we can conclude that no model exists, and the theories are inconsistent.
%\bigskip
%
%Remarks:
%\item There are some strong syntactic criteria when the Herbrand model exists. The most important one is the following. For Horn theories, the Herbrand model always exists. (In a Horn theory, all formulas in $\Theta$ must be of the form $\forall x_1\;\ldots\;\forall x_m\;((A_1\wedge \ldots A_n)\arr A)$ for atoms $A_1,\ldots,A_n,A$. Horn theories are already in Skolem normal form.)
%\item The Herbrand model of Horn theories is the base of the Prolog programming language. Prolog interprets a command $A:-A_1,\ldots,A_m$ with free variables $x_1,\ldots,x_n$ as the axiom $\forall x_1\;\ldots\;\forall x_m\;((A_1\wedge \ldots A_n)\arr A)$. A query $A$ (without free variables) is answered with Yes or No depending on whether $A$ holds in the Herbrand model constructed from all commands of the current program.
%\end{itemize}


\section{Incompleteness}\label{sec:fol:incomplete}

In Ex.~\ref{ex:nat:incomplete}, we already saw that we cannot specify the natural numbers using Peano arithmetic in FOLEQ.
We can now generalize this result.

\begin{theorem}\label{thm:natincomplete}
The set $SN^*$ from Ex.~\ref{ex:nat:incomplete} is not recursively enumerable.
\end{theorem}
\begin{proof}
Omitted.
\end{proof}

We immediately get the following corollary with far-reaching consequences:

\begin{theorem}[G\"odel's First Incompleteness Theorem]\label{thm:goedelincomplete}
There is no logic in which there is a P-consistent theory whose
  \begin{itemize}
    \item signature includes the symbols of Peano arithmetic,
    \item whose theorems include $SN^*$,
    \item whose axioms are recursively enumerable.
  \end{itemize}
\end{theorem}
\begin{proof}
If there were such a logic and such a theory, we could recursively enumerate all its theorems and thus $SN^*$.
\end{proof}

This is one of the most famous results of computer science.
It was discovered by G\"odel in 1930 \cite{goedelincompleteness}.
At around that time mathematicians were driven by Hilbert's program \cite{hilbertsprogram} to find a logic that can prove all theorems of mathematics.
G\"odel's incompleteness result showed that is not even possible to find a theory that can prove all theorems about the natural numbers (unless the theory is inconsistent, of course).
Since many data types somehow contain the natural numbers, it implies a lot of negative results for other data types.

We can understand this as a special case of the philosophical map-territory problem: A map is different from the mapped territory.
In some sense, the best map would have scale $1:1$ and be an exact copy of the territory; but that is impossible or at least impractical.
Similarly, a theory is different from the intended model, and using an exact copy of the model would be impossible or impractical.
\bigskip

G\"odel's result also imply a second incompleteness theorem: No matter what logic and what theory we use for mathematics, we will never be able to prove that the theory is consistent.
In other words, no language can prove its own consistency.
Consistency must be always be proved in a stronger language.
But then we do not know whether that stronger language is consistent, and so on.

Therefore, it is theoretically possible that tomorrow someone will discover that set theory (see Ex.~\ref{ex:set}) is inconsistent.
And then all of mathematics would break down.
This is essentially what happened in 1897 and 1901, when Peano and Russell found inconsistencies in the theories used for mathematics.
These theories had been mainly formalized by Frege in 1879 \cite{begriffsschrift} and Cantor in 1883 \cite{grundlagen} and had been used informally by all mathematicians before.

This gave rise to the development of modern set theories whose axioms are chosen very carefully to make inconsistency unlikely.
The currently used variants of set theory have been developed mainly between 1900 and 1930, but mathematicians are still contributing important details.
Parallel to the introduction of these new theories, researchers started studying logic and consistency in more detail.
Then G\"odel's result crushed all hope to settle the consistency question permanently.

