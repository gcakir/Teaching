The prototypical example of induction is mathematical induction, i.e., induction on the natural numbers.
In computer science, induction is a more general principle about ``doing something exactly once for all elements of a given set $A$''.
The ``something'' can be either of two things: a definition or a proof.

In a definition by induction, we define the function value $f(a)$ for all $a\in A$ by case distinction on $a$.
Then we use the induction principle to argue that $f$ is well-defined, i.e., that each $a\in A$ is covered by exactly one case.

In a proof by induction, we prove the property $P(a)$ for all $a\in A$ by case distinction on $a$.
Then we use the induction principle to argue that $P$ is proved, i.e., that each element of $A$ is covered by exactly one case.

\section{Mathematical Induction}\label{sec:induction:math}

The induction principle for the natural numbers $\N$ rests on the following:

\begin{definition}[Natural Numbers]\label{def:nat}
$\N$ is defined as follows:
\begin{enumerate}
  \item\label{def:nat:zero} $0\in \N$
  \item\label{def:nat:succ} if $n\in \N$, then $s(n)\in \N$ (We call $s(n)$ the successor of $n$, e.g., $s(0)=1$.)
  \item The natural numbers are exactly the ones constructed by the above two cases, i.e.,
    \begin{enumerate}
      \item\label{def:nat:consistent} All objects obtained by the cases are different: $0\neq s(n)$ for any $n$, and $s(n)\neq s(n')$ for any $n\neq n'$.
      \item\label{def:nat:coverage} Every $n\in\N$ is obtained by one the cases: for every $n\in\N$, we have $n=0$ or $n=s(n')$ for some $n'\in \N$.
   \end{enumerate}
\end{enumerate}
\end{definition}

\paragraph{Inductive Definition}
Consequently, we can define a function $f\in B^\N$ by giving $f(n)\in B$ by induction on $n\in\N$.
We have to give two cases
 \begin{itemize}
   \item a case $f(0)\in B$,
   \item a case $f(s(n))$ where we may use the value $f(n)\in B$.
 \end{itemize}

More formally, we have:
\begin{theorem}[Definition by Induction]
Given a value $V_0\in B$ and a function $V_s\in B^B$, the function $f\in B^\N$ given by
 \[f(0)=V_0\]
 \[f(s(n))=V_s(f(n))\]
is well-defined.
\end{theorem}

%\begin{example}[Factorial]
%To define $n!$, we use the cases
%  \[0!=1\]
%  \[s(n)!=n!*s(n)\]
%\end{example}

\begin{example}[Addition]\label{ex:induction:add}
We define $m+n$ by induction on $m$.

Technically, this means we define a function $f$ from $\N$ to $B=\N^\N$.
$f$ is the curried version of addition, i.e., we put $m+n=f(m)(n)$.

The cases for $f$ are
  \[f(0) \tb=\tb n\mapsto n\]
  \[f(s(m)) \tb=\tb n\mapsto s(f(m)(n))\]

A more intuitive way to write these cases is:
  \[f(0)(n) = 0+n = n\]
  \[f(s(m))(n) = s(m)+n = s(m+n)\]
\end{example}

\begin{example}[Multiplication]\label{ex:induction:mult}
We define $m\cdot n$ by induction on $m$.

Left as an exercise.
\end{example}

\paragraph{Inductive Proof}
Similarly, we can prove a property $P$ by proving $P(n)$ by induction on $n\in\N$.
We have to give two cases
 \begin{itemize}
   \item a proof of $P(0)$,
   \item a proof of $P(s(n))$ where we may assume that $P(n)$ holds.
 \end{itemize}

More formally, we have:
\begin{theorem}[Proof by Induction]
Assume $P(0)$ and ``for all $n\in\N$, if $P(n)$ then $P(s(n))$''.
Then $P(n)$ for all $n\in\N$.
\end{theorem}

\begin{example}[Addition from the Right]
We prove $m+0=m$ (AddZeroRight) by induction on $m$:
\begin{itemize}
  \item Case $m=0$. We have to prove $0+0=0$, which holds by applying the definition of $+$.
  \item Case $m=s(m')$. We have to prove that if $m'+0=m'$ (IH), then $s(m')+0=s(m')$.
   We prove this by
    \[s(m')+0=^{\mathrm{def}}s(m'+0)=^{IH}s(m')\]
   where $\mathrm{def}$ refers to the definition of $+$.
\end{itemize}

Similarly, we can show that $m+s(n)=s(m+n)$ (AddSuccRight) by induction on $m$:
\begin{itemize}
  \item Case $m=0$. We have to prove $0+s(n)=s(0+n)$, which holds by applying the definition of $+$ twice.
  \item Case $m=s(m')$. We have to prove that if $m'+s(n)=s(m'+n)$ (IH), then $s(m')+s(n)=s(s(m')+n)$.
   We prove this by
    \[s(m')+s(n)=^{\mathrm{def}}s(m'+s(n))=^{IH}s(s(m'+n))\]
   and
   \[s(s(m')+n)=^{\mathrm{def}}s(s(m'+n))\]
\end{itemize}
\end{example}

\begin{example}[Commutativity of Addition]
We prove commutativity of addition $m+n=n+m$ by induction on $m$.
Left as an exercise.
\end{example}

\begin{example}[Associativity of Addition]
We prove associativity of addition $l+(m+n)=(l+m)+n$ by induction on $m$.
\begin{itemize}
 \item Case $m=0$. We have to prove $l+(0+n)=(l+0)+n$.
   We prove this by
    \[l+(0+n)=^{\mathrm{def}}l+n\]
   and
   \[(l+0)+n=^{\mathrm{AddSuccRight}}l+n\]
 \item Case $m=s(m')$. We have to prove that if $l+(m'+n)=(l+m')+n$ (IH) then $l+(s(m')+n)=(l+s(m'))+n$.
   We prove it as follows:
    \[l+(s(m')+n)=^{\mathrm{def}} l+s(m'+n) =^{\mathrm{AddSuccRight}} s(l+(m'+n))\]
   and
    \[(l+s(m'))+n=^{\mathrm{AddSuccRight}} s(l+m')+n =^{\mathrm{def}} s((l+m')+n)\]
   and these two are equal due to (IH).
\end{itemize}
\end{example}

\begin{example}[Multiplication from the Right]
We prove $m\cdot 0=0$ (MultZeroRight) by induction on $m$.

Similarly, we can show that $m\cdot s(n)=m\cdot n+m$ (MultSuccRight) by induction on $m$.

Left as an exercise.
\end{example}

\begin{example}[Commutativity of Multiplication]
We prove commutativity of addition $m\cdot n=n\cdot m$ by induction on $m$.

Left as an exercise.
\end{example}

\begin{example}[Distributivity of Multiplication over Addition]
We prove left-distributivity $l\cdot (m+n)=l\cdot m+l\cdot n$ by induction on $l$.

Left as an exercise.

Then we immediately have right-distributivity $(m+n)\cdot l=m\cdot l+n\cdot l$ using left-distributivity and commutativity.
\end{example}

\begin{example}[Associativity of Multiplication]
We prove associativity of multiplication $l\cdot (m\cdot n)=(l\cdot m)\cdot n$ by induction on $m$.

Left as an exercise.
\end{example}

\begin{example}[Neutral Element of Multiplication]
We put $1=s(0)$ and prove $1\cdot m=m$ and $m\cdot 1=m$ (no induction necessary).

Left as an exercise.
\end{example}

\section{Regular Induction}\label{sec:induction:reg}

Given an alphabet $\Sigma$, we have the following induction principle for the set $\Sigma^*$:
\begin{definition}
$\Sigma^*$ is defined by
 \begin{itemize}
   \item $\epsilon\in\Sigma^*$,
   \item for every $x\in\Sigma$: if $w\in\Sigma^*$, then $xw\in\Sigma^*$,
   \item The elements of $\Sigma^*$ are exactly the objects obtained by the above cases.
 \end{itemize}
\end{definition}

Note that the number of cases is $|\Sigma|+1$.

\paragraph{Inductive Definition}

We give some examples for inductive definitions:

\begin{example}[Reversal]\label{ex:induction:reverse}
We define the reversal $w^R$ by induction on $w$
 \[\epsilon^R=\epsilon\]
 \[(xw)^R=w^Rx\]
\end{example}

\begin{example}[Length]
We define the length $|w|$ by induction on $w$
 \[|\epsilon|=0\]
 \[|xw|=s(|w|)\]
\end{example}

\begin{example}[DFA Transition Function]
Given a DFA with transition function $\delta(q,x)\in Q$, we define $\delta^*(q,w)$ by induction on $w$:
 \[\delta^*(q,\epsilon)=q\]
 \[\delta^*(q,xw)=\delta^*(\delta(q,x),w)\]
\end{example}

\paragraph{Inductive Proof}
We give some examples for inductive proofs:

\begin{example}[Reversal with Symbols on the Right]
We prove $(wy)^R=yw^R$ for $y\in\Sigma$ by induction on $w$.
\begin{itemize}
 \item Case $w=\epsilon$: We have to prove $(\epsilon y)^R=y\epsilon^R$. This holds because both sides are equal to $y$.
 \item Case $w=xw'$. We have to prove that if $(w'y)^R=yw'^R$, then $(xw'y)^R=y(xw')^R$.
  We prove it as follows:
    \[(xw'y)^R=^{\mathrm{def}}(w'y)^Rx =^{IH} yw'^Rx\]
  and
    \[y(xw')^R=^{\mathrm{def}} yw'^Rx\]
\end{itemize}
\end{example}

\begin{example}[Length with Symbols on the Right]
We prove $|wx|=s(|w|)$ (LengthRight) by induction on $w$.

Left as an exercise.
\end{example}

\begin{example}[Length-Preservation of Reversal]
We prove $|w^R|=|w|$ by induction on $w$.

Left as an exercise.
\end{example}

\begin{example}[Self-Inverseness of Reversal]\label{ex:induction:thm:reverse}
We prove $(w^R)^R=w$ by induction on $w$.
\begin{itemize}
 \item Case $w=\epsilon$: We have to prove $(\epsilon^R)^R=\epsilon$, which we obtain by applying the definition of reversal twice.
 \item Case $w=xw'$. We have to prove that if $(w'^R)^R=w'$ (IH), then $((xw')^R)^R=xw'$.
   This is left as an exercise.
\end{itemize}
\end{example}

\paragraph{Regular Induction as a Special Case of Mathematical Induction}
Computer science prefers using induction on $\Sigma^*$ as a stand-alone principle.

Mathematics prefers reducing it to induction on natural numbers.
Then induction on the words $w\in\Sigma^*$ becomes induction on the length $l=|w|$ of $w$.
The case $l=0$ corresponds to the case $w=\epsilon$.
And the case $l=s(l')$ corresponds to the case $w=xw'$.

\section{Context-Free Induction}\label{sec:induction:cf}

Given an unambiguous context-free grammar $(V,\Sigma,P,S)$, the context-free induction principle rests on the following:

\begin{definition}[Produced Language]\label{def:ind:cf}
For every $A\in V$, the set of words $L(A)\sq\Sigma^*$ is defined by
 \begin{itemize}
   \item For every production $A\to w_0A_1w_1\ldots w_{n-1}A_n w_n \;\in\;P$ (where $A_i\in V$ and $w_i\in\Sigma^*$): if $v_i\in L(A_i)$, then $w_0 v_1 w_1\ldots w_{n-1}v_n w_n\in L(A)$,
   \item The words in $L(A)$ are exactly the ones obtained by the above cases.
 \end{itemize}
\end{definition}

\begin{remark}
Note that the context-free induction principle defines finitely many sets at the same time: $L(A)$ for every $A\in V$.
Typically we are only interested in $L(S)$ where $S$ is the start symbol.
But it is not possible to define only $L(S)$ --- the definition of $L(S)$ must refer to $L(A)$ for other non-terminals $A$.
Therefore, we define $L(A)$ for all $A\in V$ together.
We speak of \emph{mutual induction}.
\end{remark}

\begin{remark}
If the grammar is ambiguous, then we cannot say ``exactly'' in Def.~\ref{def:ind:cf}.
The cases still cover all words, but some words are covered multiple times.
This is harmless for proofs by induction (proving something twice is harmless) but illegal for definition by induction (defining something twice is only allowed if both definitions are the same).

Therefore, we require an unambiguous grammar.
\end{remark}

\paragraph{Inductive Definition}
Given a function
  \[f\in\Set^V\]
which maps every non-terminal symbol to a set, we define
a set of functions $f_A:f(A)^{L(A)}$, i.e., one function for every non-terminal symbol $A$.
Each function $f_A$ maps the words $w\in L(A)$ produced from $A$ to elements of the set $f(A)$.

The cases of a mutually inductive definition of the $f_A$ are
 \begin{itemize}
  \item for every production $A\to w_0A_1w_1\ldots w_{n-1}A_n w_n \;\in\;P$: a value
     $f_A(w_0 v_1 w_1 \ldots w_{n-1} v_n w_n)\in f(A)$ where we may use the values $f_{A_i}(v_i)\in f(A_i)$.
 \end{itemize}

\begin{example}[Semantics of Propositional Logic]
Def.~\ref{semantics:pl} is a simple example.
Here $\FORM$ is the only non-terminal symbol, so we only define one function $\sem{-}=f_{\FORM}$ from $L(\FORM)$ to $f(\FORM)=\{0,1\}$.
\end{example}

\begin{example}[Free Variables]
Def.~\ref{def:fol:fv} uses context-free induction to define $\FV$.
There are two non-terminals and both $\FV(\TERM)=\FV(\FORM)=\pwr(\mathrm{Var})$ where $\mathrm{Var}$ is the set of possible variable names.
Then Def.~\ref{def:fol:fv} gives one case per production.
\end{example}

\begin{example}[Serialization]
The serialization of Bonus Exercise 1 is a typical example.

Here for every $A\in V$, $f(A)$ is the set of strings.
The functions $f_A$ map every word $w\in L(A)$ to its string representation.
\end{example}

\begin{example}[Regular Induction]
If we specialize to a right-linear grammar and require $f(A)$ to be the same set for all $A\in V$, we obtain regular induction as a special case.
\end{example}

\begin{example}[Mathematical Induction]\label{ex:induction:nat}
We obtain mathematical induction on the natural numbers as the special case of context-free induction on the grammar
\[N \tb ::= \tb 0\tb|\tb\suc(N)\]
\end{example}

\paragraph{Inductive Proofs}
Inductive proofs proceed very similarly.
We do not give details here.
However, keep in mind that all examples from Sect.~\ref{sec:induction:math} are examples of context-free induction via Ex.~\ref{ex:induction:nat}.

\paragraph{Context-Free Induction as a Special Case of Mathematical Induction}
Like regular induction, we can reduce context-free induction to induction on the natural numbers.
The key idea is to use induction on the length of the derivation of a word.
Alternatively, we can use induction on the height (= length of the longest branch) of the parse tree.

\section{Context-Sensitive Induction}\label{sec:induction:cs}

Context-sensitive induction is a generalization of context-free induction.
Instead of functions $f_A\in f(A)^{L(A)}$, we use functions $f_A\in f(A)^{\mathrm{Cont}\times L(A)}$ where $\mathrm{Cont}$ is the context.

The context may be different for every inductive definition.
Typically, it is a list of identifiers that have been declared together with additional information about every identifier.
The context changes during the induction; in particular, declarations are added to the context when they are encountered.

\begin{remark}
Usually, we are only interested in the function $f_S(\es,w)$ where $S$ is the start symbol and $\es$ the empty context.
But just like context-free induction requires a function for every non-terminal, context-sensitive induction additionally requires taking an arbitrary context as an argument.
\end{remark}

\paragraph{Inductive Definition}

Context-sensitive inductive definitions abound in theoretical computer science.
Many non-trivial operations --- e.g., compiling a program --- are implemented in this way (even if that is not always obvious or mentioned explicitly).

\begin{example}[Substitution]
Def.~\ref{def:fol:subapp} uses context-sensitive induction to define substitution application.

The induction context is a FOL-substitution $\gamma$, which stores which variables have been declared and which term each variable is substituted with.

$f(\TERM)$ is the set of terms, and $f(\FORM)$ is the set of formulas.
The functions $f_{\TERM}(\gamma,t)$ and $f_{\FORM}(\gamma,F)$ are written $\ov{\gamma}(t)$ and $\ov{\gamma}(F)$.

Def.~\ref{def:fol:subapp} gives one case per production.
Note how the context is used in (and only in) the case $\ov{\gamma}(x)$.
\end{example}

Inference systems are the prototypical example of context-sensitive induction:

\begin{example}[Well-Formed Syntax]
Def.~\ref{def:fol:syntax2} uses context-sensitive induction to define the well-formed terms and formulas for a fixed signature $\Sigma$.

The induction context is a FOL-context $\Gamma$, which stores which variables have been declared.

The sets $f(\TERM)$ and $f(\FORM)$ contain boolean truth values, i.e., both $f_{\TERM}$ and $f_{\FORM}$ return either ``holds'' or ``does not hold''.
We write $\istermSG{t}$ or $\isformSG{F}$ if $f_{\TERM}(\Gamma,t)$ holds or $f_{\FORM}(\Gamma,F)$ holds, respectively.

The inference rules of Fig.~\ref{fig:fol:syntax} give one case per production.
Note how the context is used in (and only in) the case $\istermSG{x}$.
\end{example}

\begin{example}[Static Analysis]
The static analysis from Bonus Exercise 1 uses context-sensitive induction.
The context stores the declared identifiers and for every identifier its type.

The return type of the functions varies: For example, $f_{\mathtt{program}}(\Gamma,E)$ is a boolean; and $f_{\mathtt{expression}}(\Gamma,E)$ is a function that takes the expected type of $E$ and returns a boolean.
\end{example}

\begin{example}[Interpretation]
The interpretation from Bonus Exercise 1 uses context-sensitive induction.
The context stores the declared identifiers and for every identifier its type and its current value.

The return type of the functions varies: $f(\mathtt{program})$ is irrelevant/empty and $f(\mathtt{expression})$ is the set containing all string, integer, and boolean values of the underlying programming language.
\end{example}

\section{Context-Free Induction in First-Order Logic}

The term language of first-order logic is great for working with context-free grammars.
In fact, we can think of FOLEQ as an abstract syntax framework for context-free grammars.

There is only one caveat: There may only be one non-terminal symbol.
However, there is a straightforward generalization of FOLEQ (called typed FOLEQ or sorted FOLEQ) that can use a separate type for each non-terminal symbol.

But the restriction to a single non-terminal is not so bad: It already covers some of the most important context-free grammars.

\begin{example}[Natural Numbers (Continuing Ex.~\ref{ex:induction:nat})]\label{ex:foldef:nat}
The grammar
\[N \tb ::= \tb 0\tb|\tb\suc(N)\]
corresponds the FOLEQ signature from Ex.~\ref{ex:nat:succ}.
\end{example}

\begin{example}[Words]\label{ex:foldef:words}
The language of words over an alphabet $A=\{a_1,\ldots,a_n\}$ can be written as the context-free grammar
\[W \tb ::= \tb \epsilon \tb|\tb a_1W \tb|\tb \ldots \tb|\tb a_nW\]
It corresponds to the FOLEQ signature using
\begin{compactitem}
\item nullary function symbol $\epsilon$
\item unary function symbols $\bm{a_1}$, \ldots, $\bm{a_n}$, written as prefix without brackets, i.e., $\bm{a}w$ instead of $\bm{a}(w)$.
\end{compactitem}
\end{example}

%\begin{example}[Regular Expressions]
%We obtain a FOL-signature for regular expressions by using:
%\begin{compactitem}
%\item nullary function symbols for the symbols of the alphabet, $\bm{\epsilon}$ and $\bm{\es}$
%\item unary function symbol $*$, written as postfix
%\item binary function symbols $+$ and $\cdot$ (concatenation), written infix
%\end{compactitem}
%\end{example}

%\begin{example}[Propositional Formulas]
%We obtain a FOL-signature for the formulas of propositional logic by using:
%\begin{compactitem}
%\item nullary function symbols for true and false,
%\item unary function symbol for negation
%\item binary function symbols for conjunction, disjunction, and implication
%\end{compactitem}
%\end{example}

\subsection{Simple Definitions in First-Order Logic}

FOLEQ does not permit inductive definitions directly.
In fact, FOLEQ signatures do not permit \emph{any} definition at all.
But we can give almost any definition indirectly using axioms in FOLEQ theories.

\paragraph{Constant Definitions}
For some term $\istermS{}{t}$, the definition $c:=t$ can be written in a FOLEQ  theory by adding
\begin{compactitem}
\item a nullary function symbol $c$,
\item the axiom $c\doteq t$
\end{compactitem}

\begin{example}[Continuing Ex.~\ref{ex:foldef:nat}]\label{ex:foldef:nat2}
We define $1:=\suc(0)$ by adding
\begin{compactitem}
\item the nullary function symbol $1$,
\item the axiom $1\doteq \suc(0)$
\end{compactitem}
\end{example}

\paragraph{Function Definitions}
For some term $\istermS{x_1,\ldots,x_n}{t}$, the definition $f$ is the function that maps $x_1,\ldots,x_n$ to $t$ can be written in a FOLEQ theory by adding
\begin{compactitem}
\item an $n$-ary function symbol $f$,
\item the axiom $\forall x_1 \ldots \forall x_n\;f(x_1,\ldots,x_n)\doteq t$
\end{compactitem}

Note that constant definitions can be seen of the special case of a function definition where $n=0$.

\begin{example}[Continuing Ex.~\ref{ex:foldef:words}]\label{ex:foldef:words2}
We define the doubling $ww$ of a word using the term $\istermS{w}{ww}$ by adding
\begin{compactitem}
\item the unary function symbol $\op{double}$,
\item the axiom $\forall w\;\op{double}(w)\doteq ww$
\end{compactitem}
\end{example}

\paragraph{Predicate Definitions}
For some formula $\isformS{x_1,\ldots,x_n}{F}$, the definition $p$ is the property that holds for $x_1,\ldots,x_n$ if $F$ holds can be written in a FOLEQ theory by adding
\begin{compactitem}
\item an $n$-ary predicate symbol $p$,
\item the axiom $\forall x_1 \ldots \forall x_n\;p(x_1,\ldots,x_n)\equiv t$
\end{compactitem}

\begin{example}[Continuing Ex.~\ref{ex:foldef:nat}]\label{ex:foldef:nat3}
We define the property $\op{nonzero}$ of a natural number using the formula $\isformS{n}{\neg n\doteq 0}$ by adding
\begin{compactitem}
\item the unary predicate symbol $\op{nonzero}$,
\item the axiom $\forall n\;\op{nonzero}(n)\equiv \neg n\doteq 0$
\end{compactitem}
\end{example}

\subsection{Inductive Definitions in First-Order Logic}

We obtain much more flexible options for definitions if we use induction.
The basic idea is the same: We use axioms to capture the content of the definition.
More precisely, we use one axiom for each case of the induction.

\paragraph{Inductive Function Definitions}
Functions are defined by adding a function symbol and then one axiom for every case.

\begin{example}[Continuing Ex.~\ref{ex:nat:arith},~\ref{ex:induction:add} and~\ref{ex:foldef:nat}]\label{ex:foldef:nat4}
We define the addition $m+n$ of natural numbers by adding
\begin{compactitem}
\item the binary function symbol $+$ (written infix),
\item the axioms $\forall n\;0+n\doteq n$ and $\forall m\forall n\;\suc(m)+n\doteq\suc(m+n)$.
\end{compactitem}

And we define the multiplication $m\cdot n$ of natural numbers by adding
\begin{compactitem}
\item the binary function symbol $\cdot$ (written infix),
\item the axioms $\forall n\;0\cdot n\doteq 0$ and $\forall m\forall n\;\suc(m)\cdot n\doteq (m\cdot n)+n$.
\end{compactitem}
\end{example}

\begin{example}[Continuing Ex.~\ref{ex:induction:reverse} and~\ref{ex:foldef:words}]\label{ex:foldef:words3}
We define the reversal of a word by adding
\begin{compactitem}
\item the unary function symbol $R$ (written as postfix superscript, i.e., $w^R$),
\item the axioms $\forall w\;\bm{\epsilon}^R\doteq \bm{\epsilon}$ and $\forall w\;(\bm{a}w)^R\doteq w^R\bm{a}$ for all unary symbols $\bm{a}$ from Ex.~\ref{ex:foldef:words}.
\end{compactitem}
\end{example}

\paragraph{Inductive Predicate Definitions}
Properties/predicates are defined by adding a predicate symbols and one axiom for every case.

\begin{example}[Continuing \ref{ex:foldef:nat}]\label{ex:foldef:nat5}
We define the less-or-equal predicate $\leq$ between natural numbers by adding
\begin{compactitem}
\item the binary predicate symbol $\leq$ (written infix),
\item the axioms \[\forall n\;0\leq n\equiv \true \tb\mand\tb \forall m\forall n\;\suc(m)\leq n\equiv\exists x\;n\doteq \suc(x)\wedge m\leq x\]
\end{compactitem}
\end{example}

\begin{remark}
Of course, we can simply write $0\leq n$ instead of $0\leq n\equiv \true$.
We use the longer version here to emphasize the general shape of inductive predicate definitions.
\end{remark}

\paragraph{Mutually Inductive Definitions}
We can also state mutually inductive definitions of functions or predicates.
We simply add multiple function/predicate symbols at once and give one axiom per case.

\begin{example}[Continuing \ref{ex:foldef:nat}]\label{ex:foldef:nat6}
We define the predicates $\op{even}$ and $\op{odd}$ on natural numbers by adding
\begin{compactitem}
\item the unary predicate symbols $\op{even}$ and $\op{odd}$,
\item the axioms \[\op{even}(0)\equiv \true \tb\mand\tb \forall n\;\op{even}(\suc(n))\equiv \op{odd}(n)\]
                 \[\op{odd}(0)\equiv \false \tb\mand\tb \forall n\;\op{odd}(\suc(n))\equiv \op{even}(n)\]
\end{compactitem}
\end{example}

\subsection{Inductive Proofs in First-Order Logic}

We can now finally understand the motivation behind the Peano axioms from Ex.~\ref{ex:nat:succ}:

\begin{example}[Induction Schemata in First-Order Logic]\label{ex:nat:fol:induction}
The Peano axioms from Ex.~\ref{ex:nat:succ} arise as follows:
\begin{itemize}
\item The function symbols $0$ and $\suc$ formalizes the cases \ref{def:nat:zero} and \ref{def:nat:succ} of Def.~\ref{def:nat}.
\item The axioms for injectivity of $\suc$ and starting point $0$ formalize the cases \ref{def:nat:consistent} of Def.~\ref{def:nat}.
  Respectively, they capture that no two successors are equal and that no successor is equal to $0$.
\item The axiom schema of induction approximately formalizes case \ref{def:nat:coverage} of Def.~\ref{def:nat}.
  It expresses the intuition that $0$ and all successors make up the whole natural numbers.
\end{itemize}
Thus, the Peano axioms systematically formalize Def.~\ref{def:nat} in FOLEQ.
\end{example}

Along these lines, we can define induction schemata in FOLEQ for all context-free grammars with a single non-terminal.
Then we can revisit the examples of inductive proofs from Sect.~\ref{sec:induction:math}, \ref{sec:induction:reg}, and~\ref{sec:induction:cf} and formally carry them out in FOLEQ.

\begin{remark}[Incompleteness of Induction]\label{ex:nat:fol:incomplete}
Note that Ex.~\ref{ex:nat:fol:induction} says ``approximately formalizes''.
The problem is that the axiom schema of induction yields only countably many axioms (because there are only countably many formulas).
But there are uncountably many things we can do with the natural numbers (e.g., uncountably many different functions out of $\N$).
Therefore, FOLEQ cannot fully capture case \ref{def:nat:coverage} of Def.~\ref{def:nat}.

Indeed, as we know from Ex.~\ref{ex:nat:incomplete}, even though the theory of Ex.~\ref{ex:nat:fol:induction} is complete (i.e., can prove or disprove every formula), there are non-standard models that satisfy the exact same FOLEQ formulas as the intended standard model $SN$.

Moreover, and even worse, we know that Peano arithmetic --- i.e., the union of the theories from Ex.~\ref{ex:foldef:nat4} and~\ref{ex:nat:fol:induction} --- is not even a complete theory:
There are formulas that are true in the standard model but can be neither proved nor disproved in FOLEQ.
The proofs of these formulas require induction hypotheses that FOLEQ's Peano arithmetic cannot express.
\end{remark}