LF does not have formulas or a consequence relation between them. Therefore, LF is not a logic. Nevertheless, LF is a type theory, and every type theory has a semantics. As for logics, there is a proof- and a model-theoretical approach to define the semantics.

\section{Model-Theoretic Semantics}

The model theory of LF can be defined very similarly to the model theory of STT.

Just like the syntax of LF needs one higher type-level than STT (namely for kinds), the model theory of LF requires one higher set-theoretical concept, namely classes. Intuitively, any collection of sets is a class. For example, $\Set$ is the class containing all sets; $\es$ is the class containing no set; every set is a class. Some classes are so ``big'' that they are not sets anymore, e.g., $\Set$ is not a set; other classes are so ``small'' that they are sets, e.g., $\es$ is a set.

Then we can interpret $\KIND$ as $\Class$ and thus kinds as classes; $\TYPE$ as $\Set$ and thus types as sets; and terms as elements of sets. We have:

\begin{center}
\begin{tabular}{|c|c|c|}
\hline
Expression          & Syntax                           & Semantics \\ \hline
kind $K$            & $\isdkind{\Sigma}{\Gamma}{K}$    & $\semm{K}{I,\alpha}\in \semm{\KIND}{}=\Class$
    \\[.2cm]
type family $A$     & $\ofdkind{\Sigma}{\Gamma}{A}{K}$  & $\semcm{\Gamma}{A}{I,\alpha}\in \semcm{\Gamma}{K}{I,\alpha}$ 
    \\[.2cm]
in particular: type $A$
                    & $\ofdkind{\Sigma}{\Gamma}{A}{\TYPE}$ & 
                               $\semcm{\Gamma}{A}{I,\alpha}\in\semcm{\Gamma}{\TYPE}{I,\alpha}=\Set$
    \\[.2cm]
term $t$            & $\ofdtype{\Sigma}{\Gamma}{t}{A}$ & $\semcm{\Gamma}{t}{I,\alpha}\in \semcm{\Gamma}{A}{I,\alpha}$ \\
\hline
\end{tabular}
\end{center}

Because the syntax of LF is defined in by mutual recursion, the same must be the case for the semantics: Models, assignments, and the interpretation of kinds, type families, and terms is defined in one big mutually recursive induction. We omit the details.

\section{Proof-Theoretic Semantics}

The proof-theoretical semantics uses additional judgments for equality of terms, type families, and kinds, and then uses an inference to axiomatize them.

The judgments are:
\begin{center}
	\begin{tabular}{|l|l|}
	  \hline
		$\isdequal{\Sigma}{\Gamma}{K}{K'}$ & $K$ and $K'$ are equal kinds over $\Sigma$ and $\Gamma$ \\
		$\isdequal{\Sigma}{\Gamma}{A}{A'}$ & $A$ and $A'$ are equal type families over $\Sigma$ and $\Gamma$ \\
		$\isdequal{\Sigma}{\Gamma}{t}{t'}$ & $t$ and $t'$ are equal terms over $\Sigma$ and $\Gamma$ \\
		\hline
	\end{tabular}
\end{center}

The equality relation must be such that:
\begin{itemize}
 \item Two kinds $A_1\arr\ldots\arr A_m \arr\TYPE$ and $A'_1\arr\ldots\arr A'_n\arr\TYPE$ can only be equal if $m=n$.
 \item Two type families can only be equal if their kinds are equal.
 \item Two terms can only be equal if their types are equal.
\end{itemize}

The inference system is given by:
\begin{itemize}
	\item reflexivity, symmetry, transitivity for terms, type families, and kinds,
	\item congruence rules for all composed expressions:
	\[\mathll[c]{
     \ibnc{\isdequal{\Sigma}{\Gamma}{A}{A'}}
          {\isdequal{\Sigma}{\Gamma}{K}{K'}}
          {\isdequal{\Sigma}{\Gamma}{A\arr K}{A'\arr K'}}
          {eq\_kdfun}
     \nl
     \ibnc{\isdequal{\Sigma}{\Gamma}{A}{A'}}
          {\isdequal{\Sigma}{\Gamma}{t}{t'}}
          {\isdequal{\Sigma}{\Gamma}{A\;t}{A'\;t'}}
          {eq\_tpapp}
     \tb\tb
     \ibnc{\isdequal{\Sigma}{\Gamma}{A}{A'}}
          {\isdequal{\Sigma}{\Gamma,x:A}{B}{B'}}
          {\isdequal{\Sigma}{\Gamma}{\P{x}{A}B}{\P{x}{A'}B'}}
          {eq\_tpfun}
     \nl     
     \ibnc{\isdequal{\Sigma}{\Gamma}{f}{f'}}
          {\isdequal{\Sigma}{\Gamma}{t}{t'}}
          {\isdequal{\Sigma}{\Gamma}{f\;t}{f'\;t'}}
          {eq\_termapp}
     \tb\tb
     \ibnc{\isdequal{\Sigma}{\Gamma}{A}{A'}}
          {\isdequal{\Sigma}{\Gamma,x:A}{t}{t'}}
          {\isdequal{\Sigma}{\Gamma}{\lam{x}{A}t}{\lam{x}{A'}t'}}
          {eq\_termlam}
	}\]
   The intuition of congruence is that composed expressions built from equal components must be equal. Note that every rule corresponds to a rule in Fig.~\ref{fig:lf:terms}.
	\item the rule
	  \[\icnc{\isdequal{\Sigma}{\Gamma}{E}{E'}}{\isdequal{\Sigma}{\Gamma}{F}{F'}}
	         {\ofdtype{\Sigma}{\Gamma}{E}{F}}{\ofdtype{\Sigma}{\Gamma}{E'}{F'}}{}\]
	   where $E,E'$ and $F,F'$ are (i) terms and types, respectively, or (ii) type families and kinds, respectively.
	   This rule says that typing respects equality. It can also be seen as a congruence rule for the typing judgment.
	\item $\beta$ and $\eta$-equality for terms:
	  \[
	     \ianc{\ofdtype{\Sigma}{\Gamma}{(\lam{x}{A}t)\;s}{B}}
            {\isdequal{\Sigma}{\Gamma}{(\lam{x}{A}t)\;s}{t[x/s]}}
            {beta}
       \tb\tb
       \ibnc{\ofdtype{\Sigma}{\Gamma}{f}{\P{x}{A}B}}
            {x\mnot\minn \Gamma}
            {\isdequal{\Sigma}{\Gamma}{f}{\lam{x}{A}(f\; x)}}
            {eta}
      \]
\end{itemize}

Like many other type theories, the equality of LF enjoys the normalization property: For every expression (i.e., every kind, type family, or term), there is a normal form; every expression is equal to its normal form, and two expressions are equal iff their normal forms are identical (up to $\alpha$-renaming of variables). Because the normal form can be computed, the equality judgments are decidable: To decide if two expressions are equal, we check if they have the same normal form.

\section{The Relation between Model and Proof Theory}

For type theories, we can define soundness and completeness as follows. Soundness: If two expressions are provably equal, then their interpretations are equal in every model and under every assignment. Completeness: If the interpretations of two expressions are equal in every model and under every assignment, then they are provably equal. In other words, soundness and completeness relate:
 \[\isdequal{\Sigma}{\Gamma}{E}{E'}  \tb\mand\tb \semcm{\Gamma}{E}{I,\alpha}=\semcm{\Gamma}{E'}{I,\alpha}\mforall I,\alpha\]
for pairs $(E,E')$ of kinds, type families, or terms.

Similarly to STT and HOL, we have soundness but not completeness for LF. To obtain completeness, we would need a larger collection of models, and the needed models are even weirder than the ones needed for STT and HOL.