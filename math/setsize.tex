The size $|S|$ of a set $S$ is a very complex operation because there are different degrees of infinity, i.e., not all infinite sets have the same size.
Specifically, we have that $|\pwr(S)|>|S|$, i.e., we have infinitely many degrees of infinity.

In computer science, we are only interested in countable sets.
Therefore, we use a much simpler definition of size: we write $C$ for \emph{countable} and $U$ for uncountable, i.e., everything that is bigger:

\begin{definition}[Size of sets]\label{def:setsize}
The size $|S|\in\N\cup\{C,U\}$ of a set $S$ is defined by:
\begin{itemize}
\item if $S$ is finite: $|S|$ is the number of elements of $S$
\item if $S$ is infinite and bijective to $\N$: $|S|=C$, and we say that $S$ is \textbf{countable}
\item if $S$ is infinite and not bijective to $\N$: $|S|=U$, and we say that $S$ is \textbf{uncountable}
\end{itemize}
\end{definition}

We can compute with set sizes as follows:

\begin{definition}[Computing with Sizes]\label{def:setsizecomp}
For two sizes $s,t\in \N\cup\{C,U\}$, we define addition, multiplication, and exponentiation by the following tables:

\begin{center}
\begin{tabular}{cc|ccc}
      &           & \multicolumn{3}{c}{$t$} \\
      & $s+t$ & $n\in \N$ & $C$ & $U$ \\
\hline
      & $m\in \N$ & $m+n$ & $C$ & $U$ \\ 
$s$   & $C$       & $C$   & $C$ & $U$ \\
      & $U$       & $U$   & $U$ & $U$
\end{tabular}
\tb\tb
\begin{tabular}{cc|ccc}
      &           & \multicolumn{3}{c}{$t$} \\
      & $s*t$     & $n\in \N$ & $C$ & $U$ \\
\hline
      & $m\in \N$ & $m*n$ & $C$ & $U$ \\ 
$s$   & $C$       & $C$   & $C$ & $U$ \\
      & $U$       & $U$   & $U$ & $U$
\end{tabular}

\begin{tabular}{cc|ccccc}
      &                  & \multicolumn{5}{c}{$t$} \\
      & $s^t$            & $0$ & $1$ & $n\in\N\sm\{0\}$ & $C$ & $U$ \\
\hline
      & $0$              & $1$ & $0$ & $0$ & $0$ & $0$ \\ 
      & $1$              & $1$ & $1$ & $1$ & $1$ & $1$ \\ 
$s$   & $m\in\N\sm\{0\}$ & $1$ & $m$ &$m^n$& $U$ & $U$ \\ 
      & $C$              & $1$ & $C$ & $C$ & $U$ & $U$ \\
      & $U$              & $1$ & $U$ & $U$ & $U$ & $U$
\end{tabular}
\end{center}
Because exponentiation $s^t$ is not commutative, the order matters: $s$ is given by the row and $t$ by the column.
\end{definition}

The intuition behind these rules is given by the following:
\begin{theorem}
For all sets $S,T$, we have for the size of the
\begin{compactitem}
 \item disjoint union: \[|S \uplus T| = |S| + |T|\]\
 \item Cartesian product: \[|S\times T| = |S| * |T|\]
 \item set of functions from $T$ to $S$: \[|S^T| = |S| ^{|T|}\]
\end{compactitem}
\end{theorem}

Thus, we can understand the rules for exponentiation as follows.
Let us first consider the $4$ cases where one of the arguments has size $0$ or $1$: For every set $A$
\begin{compactenum}
\item there is exactly one function from the empty set (namely the empty function): $|A^\es|=1$,
\item there are as many functions from a singleton set as there are elements of $A$: $|A^{\{x\}}|=|A|$,
\item there are no functions to the empty set (unless $A$ is empty): $|\es^A|=0$ if $A\neq\es$,
\item there is exactly one function into a singleton set (namely the constant function): $|\{x\}^A|=1$,
\end{compactenum}

Now we need only one more rule: The set of functions from a non-empty finite set to a finite/countable/uncountable set is again finite/countable/uncountable.
In all other cases, the set of functions is uncountable.
